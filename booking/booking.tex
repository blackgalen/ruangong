\documentclass[11pt]{article}

%减少左右空白边缘
%\usepackage{fullpage}
\usepackage[left=1.3in, right=1.3in, top=1.3in, bottom=1.3in]{geometry}
%插入图像 .jpg .png .pdf
\usepackage{graphicx}
%支持中文
\usepackage[UTF8, heading = false, scheme = plain]{ctex}
% 页眉页脚
\usepackage{fancyhdr}
\pagestyle{fancy}
\fancyhead{}
\fancyfoot{}
\fancyhead[L]{\slshape Left Title}
\fancyhead[R]{\slshape Right Title}
\fancyfoot[C]{\thepage}

\begin{document}

% 标题页
\begin{titlepage}
\begin{center}

	\huge{\textbf{软件工程\ 课程设计报告}}\\

	\vfill
	\line(1, 0){400} \\ [1mm]
	\large{\textbf{09015116\ 樊博杰}} \\ [1mm]
	\large{\textbf{09015124\ 邹家伦}} \\ [1mm]
	\large{\textbf{09015128\ 张敏学}} \\ [1mm]
	\large{\textbf{09015204\ 杨艳春}} \\ [1mm]
	\large{\textbf{09015209\ 刘韦}} \\ [1mm]
	\line(1, 0){400} \\
	\large{\today}\\
\end{center}
\end{titlepage}

%目录
\tableofcontents
\thispagestyle{empty}
\clearpage


\setcounter{page}{1}
\fancyhead[R]{\slshape 用例与需求分析}
\section{用例与需求分析}
   缤客(Booking.com)是一家可以帮用户在网上预订世界各地住宿的网站。向用户提供各种类型住宿最优惠的价格,其中既有小型的家庭经营住宿加早餐旅馆,也有五星级豪华酒店。致力于提供信息全面且易于使用的网站以及最优惠价格保证。目标是为世界各地的商务人士和休闲旅客提供最易于使用且最实惠的方式,搜索并预订各种类型的住宿。公司的多语种客户服务团队致力于为所有的客人提供协助和支持。
	\subsection{参与者}
	一共有三类参与者: 酒店合作伙伴(Partner), 管理员(Administrator), 以及顾客(User)
	\subsection{分销合作伙伴 Partner}
		针对分销合作伙伴,booking网站提供了注册登录,访问酒店后台,更新日历,房价及其他设置,上线接受预定等功能。
		\subsubsection{注册登录}
		booking提供的注册功能,可以让用户注册成为分销合作伙伴,用户可以用邮箱注册,完成注册后,将审核用户提交的相关信息,确保信息正确完整。一旦通过审核,用户将收到邮件,内附“酒店后台”的登录信息。随后,合作伙伴还将收到操作指南
		\subsubsection{访问酒店后台}
		“酒店后台”是为合作伙伴提供的住宿管理系统。合作伙伴可用于编辑修改房量、房价等信息。
		\subsubsection{上线接受预定}
		完成注册审核后,会发送邮件,告知贵住宿在Booking.com上线的后续步骤。订单即时确认
        	所有来自Booking.com的客人订单均为即时确认,为酒店合作伙伴省心省力。
        	\subsubsection{24小时客户支持}
        	当地支持团队全天24小时为酒店提供协助,支持41种语言。
        	\subsubsection{酒店评论查看与回复}
        	评语审核团队将对所有客人评语进行核查,只有客人实际入住了预订的酒店,提交的评语方可显示在网站上。这样的评语更有可信度,能够为酒店吸引更多客人。合作伙伴可以查看有关酒店的评语,并进行回复。
	\subsection{管理员 Administrator}
        针对 BOOKING 网站管理员,BOOKING 主要提供了登录,管理用户,提供推送,在线客服等主要功能。 
        	\subsubsection{登陆}
        	管理员可以凭借管理员账号登陆 BOOKING 网站,获得管理权限。 
        	\subsubsection{管理用户}  
        	管理员可以审核酒店合作伙伴的注册信息,通过或拒绝申请。也可以对行为不当的用户进行删除或者封号。对注册成功的酒店合作伙伴账号进行删除或封号。 
        	\subsubsection{提供推送}
        	管理员可以根据热门的旅游地点,旅游主题,优惠折扣等发布推荐,更新常见问题列表,发布最新 新闻,发布招聘信息等一系列的推送。 
        	\subsubsection{在线客服}  
        	管理员可以查看来自酒店合作伙伴和普通用户的建议或提问,可以给出回复,并且可以查看用户对在线客服的评价。
		\subsubsection{订单修改} 
		管理员可按用户的需求帮助用户修改订单。
		
	\subsection{顾客 User}
		针对普通用户, booking 网站提供了注册登录, 酒店搜索和预订, 评论, 设置等功能
		\subsubsection{注册登录}
		booking 提供的注册功能可以让游客注册成为用户,用户可以用已有账号登录,微信登录,或绑定其他的社交账号,例如 Google,Facebook 等。
		\subsubsection{查看推送}
		booking 会根据热门的旅游主题,旅游地点,优惠折扣,用户所在地等推荐相关的酒店,并且提供详细的酒店信息。用户可以随时查看最热门的推送消息,选择合适的酒店。
		\subsubsection{酒店搜索}
		booking 提供的搜索功能可以根据目的地、住宿名称或地址,针对住宿时间、出行类别、客房数量、随行人数等查询有空余房间的符合条件的酒店。booking 非常贴心的根据全球驴友的出行记录和喜好为用户提供了根据旅行关键词和地点搜索受欢迎的酒店。在通过目的地搜索时,用户可以选择使用地图浏览模式,更直观便捷。搜索完成后,booking 系统优先显示热门推荐酒店和网友推荐旅行主题,用户可以选择按价格、评分、距离市中心远近、星级等进行排序。并且可以通过一些筛选类别来缩小搜索范围,筛选类别有:每晚房价、热门筛选、住宿房量、住宿评级、住处类型、设施、餐点、预定政策、优惠、评分、床型偏好、24 小时前台、热门主题/活动、海滩、客房设施、酒店所在区、连锁酒店。针对用户选择的酒店,具体提供了以下信息:酒店星级、酒店价格、酒店照片、相关文字介绍、用户评论、在地图上的位置、推荐的精彩活动、可预订空房列表、设施、相关条款、预订须知等,并会根据用户点评提炼出关键词。用户可以选择咨询客服、将酒店添加至心愿单、立刻预订、分享给好友。
		\subsubsection{酒店预订}
			用户搜索到符合条件的酒店,网站会展示酒店信息的图文及游客的评价,供用户全面了解酒店的信息;若决定立即预定,用户将要填写自己的个人信息和自己的进一步详细需求,输入银行卡号,用户暂时无需付款。但住宿提供方可能会暂时冻结银行卡内的部分金额,以测试银行卡有效性并担保订单。这部分金额稍后将会退还。最后完成预定。
		
		\subsubsection{浏览记录}
		用户在浏览 booking 网站时,网站会提供用户的最新浏览记录,浏览该酒店的人数,最新的用户评论,最新订单等实时信息。
		
        \subsubsection{评语}
        用户可以对入住的酒店进行评价,或推荐,也可参与旅行问答,编写的评语将会转至待审核的评语栏等待工作人员审核。
        \subsubsection{我的收藏}
        用户可以将心仪的住宿保存至收藏列表,方便之后查找;也可分享自己的收藏那个,或创建新的心愿单。
        \subsubsection{个人设置}
        用户可在个人设置中完善自己的个人信息,预订信息和银行卡信息;进行支付设置,选择不同的支付方式;用户也可设置旅游偏好,便于网站提供给用户个性化的推荐。
		
		




\pagebreak
\fancyhead[R]{\slshape 用例图, 事件流, 类图, 顺序图, 状态图}
\section{用例图, 事件流, 类图, 顺序图, 状态图}



	\subsection{用户管理}
	
		\subsubsection{注册}
			\begin{tabular}{c|l}
			\hline
			用例名称 & 注册Register \\ \hline
			参与者 & 酒店合作伙伴Partner, 顾客User  \\ \hline
			入口条件 & 用户点击注册按钮 \\ \hline
			事件流 & 	\parbox{33em}{\ \\
						1. 用户点击注册按钮 \\
						2. 系统弹出注册窗口, 用户在窗口中输入账号, 密码, 确认密码, 邮箱, 手机号等信息, 选择注册类型(Partner, User), 点击确认  \\
						3. 系统弹出验证码输入窗口 \\
						4. 用户点击获取验证码, 系统向手机号或邮箱发送验证码 \\
						5. 用户输入验证码, 点击确认 \\
						6. 注册成功. 系统提示注册成功, 并向邮箱发送注册成功的邮件 \\
						} \\ \hline
			出口条件 & 注册成功或用户主动退出 \\ \hline
			质量需求 & \parbox{33em}{\ \\
						1. 用户两次输入的密码相匹配, 邮箱, 手机号格式正确 \\ 
						2. 用户名未被注册 \\
						3. 验证码须在10min之内成功验证 \\
						} \\ \hline
			\end{tabular}



		\subsubsection{登录}
			\begin{tabular}{c|l}
			\hline
			用例名称 & 登录 login \\ \hline
			参与者 & 酒店合作伙伴Partner, 管理员Administrator, 以及顾客User  \\ \hline
			入口条件 & 用户点击登录按钮 \\ \hline
			事件流 & 	\parbox{33em}{\ \\
						1. 用户点击登录按钮 \\
						2. 系统弹出登录窗口 \\
						3. 用户在窗口中输入账号, 密码和验证码, 点击确认  \\
						4. 系统在后台验证用户输入的账号, 密码以及验证码的正确性 \\
						5. 登录成功 \\
						} \\ \hline
			出口条件 & 登录成功或用户主动退出 \\ \hline
			质量需求 & \parbox{33em}{\ \\
						1. 用户输入的账号和密码相匹配 \\
						2. 用户输入的验证码正确 \\
						} \\ \hline
			\end{tabular}\\ \\ \\ \\ 
			
						
			实体对象 \\
			\begin{tabular}{l|l}\hline
			用户 User & 这是希望登入系统的用户 \\ \hline
			用户数据库 User DB & 存储用户信息的数据库 \\ \hline
			\end{tabular} \\ \\ \\ \\


			边界对象 \\
			\begin{tabular}{l|l}\hline
			登录按钮 Login Button & 登录按钮 \\ \hline
			登录界面 Login Page & 登录界面 \\ \hline
			\end{tabular} \\ \\ \\ \\

			
			控制对象 \\
			\begin{tabular}{l|l}\hline
			登录界面控件 Login Control & 用来控制登录界面 \\ \hline
			登录模型 Login Model & 用来判断用户信息 \\ \hline 	
			数据库控件 DB Controller & 用来控制数据库 \\ \hline
			\end{tabular} \\ \\ \\ \\

		
		\subsubsection{登出}
		
			\begin{tabular}{c|l}
			\hline
			用例名称 & 登出 logout \\ \hline
			参与者 & 酒店合作伙伴Partner, 管理员Administrator, 以及顾客User  \\ \hline
			入口条件 & 用户点击退出按钮 \\ \hline
			事件流 & 	\parbox{33em}{\ \\
						1. 用户点击退出按钮 \\
						2. 系统弹出确认退出窗口 \\
						3. 用户在窗口中点击确认  \\
						4. 系统在后台退出用户的登录状态 \\
						5. 退出成功 \\
						} \\ \hline
			出口条件 & 退出成功 \\ \hline
			质量需求 & \parbox{33em}{\ \\
						用户在确认退出窗口中点击确认 \\
						} \\ \hline
			\end{tabular}\\ \\ \\ \\ 
			
		\subsubsection{密码修改}
			\begin{tabular}{c|l}
			\hline
			用例名称 & 密码修改 Change Password \\ \hline
			参与者 & 酒店合作伙伴Partner, 管理员Administrator, 以及顾客User  \\ \hline
			入口条件 & 用户点击修改密码按钮 \\ \hline
			事件流 & 	\parbox{33em}{\ \\
						1. 用户点击修改密码按钮 \\
						2. 系统弹出修改密码页面, 页面中包含: 新密码, 新密码确认, 验证码, 获取验证码的按钮, 确认按钮以及取消按钮\\
						3. 用户在页面中输入相关信息, 点击确认  \\
						4. 系统判断两次输入密码是否一致, 验证码是否正确 \\
						5. 密码修改成功 \\
						} \\ \hline
			出口条件 & 密码修改成功 \\ \hline
			质量需求 & \parbox{33em}{\ \\
						1. 用户两次输入的密码一致 \\
						2. 用户输入的验证码正确 \\
						} \\ \hline
			\end{tabular}\\ \\ \\ \\ 

		\subsubsection{用户信息修改}
			\begin{tabular}{c|l}
			\hline
			用例名称 & 用户信息修改 Change Information \\ \hline
			参与者 & 酒店合作伙伴Partner, 管理员Administrator, 以及顾客User  \\ \hline
			入口条件 & 用户点击修改用户信息按钮 \\ \hline
			事件流 & 	\parbox{33em}{\ \\
						1. 用户点击修改用户信息按钮 \\
						2. 系统弹出修改用户信息页面, 页面中包含: 用户名, 手机号, 邮箱, , 验证码, 获取验证码的按钮, 确认按钮以及取消按钮\\
						3. 用户在页面中输入相关信息, 点击确认  \\
						4. 系统判断新手机号和邮箱是否合法, 验证码是否正确 \\
						5. 密码修改成功 \\
						} \\ \hline
			出口条件 & 用户信息修改成功 \\ \hline
			质量需求 & \parbox{33em}{\ \\
						1. 用户输入的手机号和邮箱合法 \\
						2. 用户输入的验证码正确 \\
						} \\ \hline
			\end{tabular}\\ \\ \\ \\ 

			
			
			
			
	\subsection{酒店合作伙伴Partner}
		\subsubsection{管理酒店信息}
		\subsubsection{查看或回复评论}
		\subsubsection{联系客服}
		\subsubsection{退出加盟}
		
	\subsection{管理员Administrator}
		\subsubsection{管理加盟申请}
		\subsubsection{管理用户}
		
		
	
	\subsection{普通用户User}
	
		\subsubsection{酒店搜索}
			\begin{tabular}{c|l}
			\hline
			用例名称 & 酒店搜索 Hotel Search \\ \hline
			参与者 & 顾客User  \\ \hline
			入口条件 & 用户在搜索界面上输入相关信息, 点击搜索按钮 \\ \hline
			事件流 & 	\parbox{33em}{\ \\
						1. 用户在搜索界面上输入城市, 入住日期,  离开日期, 点击搜索按钮 \\
						2. 系统获取用户的输入信息, 在数据库中查找符合条件的酒店 \\
						3. 系统将符合条件的酒店显示到界面上, 并在地图上标注  \\
						} \\ \hline
			出口条件 & 系统显示搜索结果或用户主动退出 \\ \hline
			质量需求 & \parbox{33em}{\ \\
						用户输入的搜索信息完整 \\
						} \\ \hline
			\end{tabular}

		\subsubsection{预订}
		\subsubsection{支付}
		\subsubsection{评论}
		\subsubsection{收藏}
		\subsubsection{订单查询}
	
	\subsection{售后服务?}
		\subsubsection{查看常见问题和解答}
		\subsubsection{查看相关条款}
		\subsubsection{联系客服}
	\subsection{软件设置}
		\subsubsection{语言选择}
		\subsubsection{主题选择}
		\subsubsection{货币选择}
	\subsection{其他服务}
		\subsubsection{租车服务}
			\begin{tabular}{c|l}
			\cline
			用例名称 & 用户信息修改 Change Information \\ \hline
			参与者 & 酒店合作伙伴Partner, 管理员Administrator, 以及顾客User  \\ \hline
			入口条件 & 用户点击修改用户信息按钮 \\ \hline
			事件流 & 	\parbox{33em}{\ \\
						1. 用户点击修改用户信息按钮 \\
						2. 系统弹出修改用户信息页面, 页面中包含: 用户名, 手机号, 邮箱, , 验证码, 获取验证码的按钮, 确认按钮以及取消按钮\\
						3. 用户在页面中输入相关信息, 点击确认  \\
						4. 系统判断新手机号和邮箱是否合法, 验证码是否正确 \\
						5. 密码修改成功 \\
						} \\ \hline
			出口条件 & 用户信息修改成功 \\ \hline
			质量需求 & \parbox{33em}{\ \\
						1. 用户输入的手机号和邮箱合法 \\
						2. 用户输入的验证码正确 \\
						} \\ \hline
			\end{tabular}\\ \\ \\ \\ 
		\subsubsection{三餐预订}
\end{document}
